\documentclass[english,xcolor=svgnames]{beamer}


\input{../Templates/teachingslidesbeamer.tex}



\begin{document}

\title{Fixed Cost Models}
\vspace{1cm}
\author[shortname]{
\begin{tabular}{cc}
Juan Herre\~{n}o & Johannes Wieland \\ 
\end{tabular}\\
}



\date{UCSD, Spring \the\year}

\setbeamertemplate{footline}{}
\makebeamertitle
\setbeamertemplate{footline}[frame number]{}

\addtocounter{framenumber}{-1}



%\begin{frame}
%\frametitle[alignment=center]{Reminders}
%\begin{enumerate}
%	\item First project draft due May 1.
%	\item Participation.
%\end{enumerate}
%\end{frame}


%%%%%%%%%%%%%%%%%%%%%%%%%%%%%%%%%%%%%%%%%%%%%%%%%%
\AtBeginSection[]{
\setbeamertemplate{footline}{}
  \frame<beamer>{ 

    \frametitle{Outline}   

    \tableofcontents[currentsection,hideallsubsections] 
  }
\setbeamertemplate{footline}[frame number]{}
\addtocounter{framenumber}{-1}
}

\AtBeginSubsection[]{
\setbeamertemplate{footline}{}
  \frame<beamer>{ 

    \frametitle{Outline}   

    \tableofcontents[currentsection,currentsubsection] 
  }
  \setbeamertemplate{footline}[frame number]{}
  \addtocounter{framenumber}{-1}
}



\setbeamertemplate{footline}{}
\begin{frame}
\frametitle{Outline}   
\tableofcontents[hideallsubsections] 
\end{frame}
\addtocounter{framenumber}{-1}
\setbeamertemplate{footline}[frame number]{}


%%%%%%%%%%%%%%%%%%%%%%%%%%%%%%%%%%%%%%%%%%%%%%%%%%
\section{Introduction}
%%%%%%%%%%%%%%%%%%%%%%%%%%%%%%%%%%%%%%%%%%%%%%%%%%


\begin{frame}
    \frametitle{Fixed Cost Models}
    \begin{itemize}
        \item Lots of behavior at the micro level is lumpy: buying houses, cars, setting prices, investment, information acquisition.
        \item Fixed cost models naturally give rise to such lumpiness. 
        \item Fixed cost models are hard to solve because they are non-convex, non-differentiable, and in GE have to carry distribution around.
        \item But lots of interesting economics: state-dependence, size-dependence, policy targeting.
        \item Today we will see how to solve these models within SSJ routines.
    \end{itemize}
\end{frame}

\begin{frame}
    \frametitle{Fixed Cost Models: Examples}
    \begin{itemize}
        \item Does lumpy price setting behavior matter? lots from Alvarez-Lippi; Auclert, Rigato, Rognlie, Straub (2024).
        \item Do fixed cost frictions matter? Kahn-Thomas, Winberry (2021), Koby-Wolf (2020), Bailey-Blanco (2021).
        \item Lumpy durables and monetary transmission: McKay and Wieland (2021, 2022).
    \end{itemize}
\end{frame}


%%%%%%%%%%%%%%%%%%%%%%%%%%%%%%%%%%%%%%%%%%%%%%%%%%
\section{Baseline FC Model}
%%%%%%%%%%%%%%%%%%%%%%%%%%%%%%%%%%%%%%%%%%%%%%%%%%

\begin{frame}
    \frametitle{Environment}
    \begin{itemize}
        \item Liquid assets $a$
        \item Durable stock $d$
        \item Nondurable consumption $c$
        \item Income $y$
        \item Real rate $r$
        \item Depreciation rate $\delta$
        \item Fixed cost $f$
    \end{itemize}
\end{frame}

\begin{frame}
    \frametitle{Constraints}
    \begin{itemize}
        \item Budget constraint if adjusting:
        \begin{align*}
            \frac{a'}{1+r} +c'+ p d' = a + (1-\delta)(1-f)pd + y \equiv x_{adj}
        \end{align*}
        \item Budget constraint if not adjusting with $d'= (1-\delta)d$
        \begin{align*}
            \frac{a'}{1+r}+c' = a + y \equiv x_{noadj}
        \end{align*}
        \item Borrowing constraint
        \begin{align*}
            a'\ge 0
        \end{align*}
    \end{itemize}
\end{frame}


\begin{frame}
    \frametitle{Value Functions}
    \begin{itemize}
        \item Let the value function be
        \begin{align*}
            V(y,d,a,\epsilon) = \max\{V^{adj}(y,d,a) + \epsilon^{adj}, V^{noadj}(y,d,a) + \epsilon^{noadj} \}
        \end{align*}
        \begin{itemize}
            \item $\epsilon^i$ are drawn from a Gumbell distribution with standard deviation $\sigma_V\frac{\pi}{\sqrt{6}}$,\footnote{https://eml.berkeley.edu/choice2/ch3.pdf}.
        \end{itemize}
        Define the post-adjustment value function as
        \begin{align*}
            W(y,d',a') \equiv E_y \tilde{V}(y',d',a')
        \end{align*}
        \item Value from adjusting and not adjusting:
        \begin{align*}
            V^{adj}(y,d,a)&=\max_{c'+d'+\frac{a'}{1+r}\le x_{adj}}[\psi \ln c' +(1-\psi) \ln d' + W(y,d',a'), \\
            V^{noadj}(y,d,a)&=\max_{c'+\frac{a'}{1+r}\le x_{noadj}}[\psi \ln c' +(1-\psi) \ln (1-\delta)d + W(y,(1-\delta)d,a')]\}
        \end{align*}
    \end{itemize}
\end{frame}

\begin{frame}
    \frametitle{Adjustment Probabilities and Expected Values}
    \begin{itemize}
        \item The probability of adjustment is
        \begin{align*}
            adjust(y,d,a) = \frac{\exp\{(V^{adj}(y,d,a)-V^{noadj}(y,d,a))/\sigma_V\}}{1+\exp\{(V^{adj}(y,d,a)-V^{noadj}(y,d,a))/\sigma_V\}}
        \end{align*}
        \item The distribution of $V$ is
        \begin{align*}
            Prob[V\le x] &= Prob[\epsilon^{adj}\le x - V^{adj}(y,d,a)] \\
            &\times Prob[\epsilon^{noadj}\le x - V^{noadj}(y,d,a)]\\
            % &= \exp\{-\exp[ -(x-V^{adj}(y,d,a))/\sigma_V]\}\exp\{-\exp[- (x-V^{noadj}(y,d,a))/\sigma_V]\} \\
            % &= \exp\{-\exp[ -(x-V^{adj}(y,d,a))/\sigma_V]-\exp[- (x-V^{noadj}(y,d,a))/\sigma_V]\} \\
            % &= \exp\{-\exp[ -x/\sigma_V](\exp[ V^{adj}(y,d,a)/\sigma_V]+\exp[V^{noadj}(y,d,a)/\sigma_V])\} \\
            &= \exp\{-\exp[ -(x - \sigma_V\log\{\exp[ V^{adj}(y,d,a)/\sigma_V]\\
            &+\exp[V^{noadj}(y,d,a)/\sigma_V]\})/\sigma_V]\} 
        \end{align*}
        \item The expected value is,
        \begin{align*}
            \tilde{V}(y,d,a) &\equiv E_{\epsilon} V(y,d,a,\epsilon) \\
            % &=\sigma_V\log\{\exp[ V^{adj}(y,d,a)/\sigma_V]+\exp[V^{noadj}(y,d,a)/\sigma_V]\} + \sigma_V\gamma \\
            % &=\sigma_V\log (\exp[ V^{noadj}(y,d,a)/\sigma_V]\{1+\exp[(V^{adj}(y,d,a)-V^{noadj}(y,d,a))/\sigma_V]\}) + \sigma_V\gamma \\
            % &=\sigma_V\{ V^{noadj}(y,d,a)/\sigma_V + \log\{1+\exp[(V^{adj}(y,d,a)-V^{noadj}(y,d,a))/\sigma_V]\} + \sigma_V\gamma \\
            % &=\sigma_V\{ V^{noadj}(y,d,a)/\sigma_V + (V^{adj}(y,d,a)-V^{noadj}(y,d,a))/\sigma_V - \log adjust(y,d,a) \} + \sigma_V\gamma\\
            % &=V^{adj}(y,d,a) - \sigma_V  \log adjust(y,d,a) + \sigma_V\gamma \\
            &=V^{noadj}(y,d,a) - \sigma_V  \log noadjust(y,d,a) + \sigma_V\gamma 
        \end{align*}
        where $\gamma\approx 0.5772$ is the Euler-Mascheroni constant. If $\sigma_V\rightarrow0$, then
        \begin{align*}
            \tilde{V}(y,d,a) &= \max\{V^{adj}(y,d,a), V^{noadj}(y,d,a) \} 
        \end{align*}
    \end{itemize}
\end{frame}


\begin{frame}
    \frametitle{No Adjustment Problem}
    \begin{itemize}
        \item Break up into sequential problem of choosing $c'$, $a'$ given $d'$. Then choose $d'$.
        \item For given choice of $d'$:
        \begin{align*}
            V^{noadj}(y,n,m) &= \max_{c',a'}[\psi \ln c' +(1-\psi) \ln d' + \beta W(y',d',a')] \\
            d' &= n \\
            a' &= \left[m - c'    \right] \\
            a' &\ge 0
        \end{align*}
        where $m \ge 0$.
        \item First order conditions:
        \begin{align*}
            V_m^{noadj}(y,n,m) &= \beta W_a(y',d',a') + \zeta \\
            V_n^{noadj}(y,n,m) &=\frac{(1-\psi)}{d'} + \beta W_d(y',d',a') \\
            \frac{\psi}{c'} &= \beta W_a(y',d',a')  + \zeta 
        \end{align*}
    \end{itemize}
\end{frame}


\begin{frame}
    \frametitle{EGM}
    \begin{itemize}
        \item We are given an initial guess $W_a(y',d',a')$.
        \item First, assume borrowing constraint is not binding and solve for $c'$ in 
        \begin{align*}
            \frac{\psi}{c'} &= \beta  W_a(y',d',a') 
        \end{align*}
        \item Then calculate implied cash on hand
        \begin{align*}
            m = c' + a' 
        \end{align*}
        \item The problem is not necessarily concave, so there may be multiple combinations of $(c',a')$ that map into the same grid point $m$.
        \begin{itemize}
            \item In this case we have to check which solution yields highest utility and discard the others.
            \item The "func\_upper\_envelop.py" function performs this task.
        \end{itemize}
    \end{itemize}
\end{frame}


\begin{frame}
    \frametitle{EGM}
    \begin{itemize}
        \item Interpolate the decision rules from the $m$ grid onto the grid for $a$.
        \item Then check if implied solution for $a'$ violates the borrowing constraint. If so implement borrowing constrained solution $a'=0$ and $c' = m$.
        \item Combining the two solutions yields the policy functions
        \begin{align*}
            c^{noadj}(y,n,m), a^{noadj}(y,n,m)
        \end{align*}
        \item The no adjustment solutions use $n=(1-\delta)d$ and $m=a +y$
        \begin{align*}
            &c^{noadj}(y,(1-\delta)d,m) \\
            &a^{noadj}(y,(1-\delta)d,m) \\
            &d^{noadj}=(1-\delta)d
        \end{align*}
    \end{itemize}
\end{frame}



\begin{frame}
    \frametitle{Adjustment Problem}
    \begin{itemize}
        \item Given some amount of cash on hand $x$:
        \begin{align*}
            V^{adj}(y,x) &= \max_{d'} V^{noadj}(y,d',m') \\
            m' &= x+y-pd'  \\
            m' &\ge 0
        \end{align*}
        \item FOC
        \begin{align*}
            V_x^{adj}(y,x) &= V_m^{noadj}(y,d',m') + \kappa \\
            V_n^{noadj}(y,d',m') &= p V_m^{noadj}(y,d',m') + \kappa
        \end{align*}
        \item Constraint will never be binding so $\kappa=0$. 
        \item Combine with keep FOC to get
        \begin{align*}
            p\frac{\psi}{c^{noadj}(y,d',m')} &=   \frac{1-\psi}{d'}  + \beta W_d(y,d',a') 
        \end{align*}
        \item Again must solve for upper envelope given non-concavity.
    \end{itemize}
\end{frame}


\begin{frame}
    \frametitle{Adjustment Solution}
    \begin{itemize}
        \item This yields adjustment solutions
        \begin{align*}
            c^{adj}(y,x), a^{adj}(y,x), d^{adj}(y,x)
        \end{align*}
        and a value function
        \begin{align*}
            V^{adj}(y,x) &= \psi \ln c^{adj}(y,x) +(1-\psi) \ln d^{adj}(y,x) \\
            &+ \beta W(y',d^{adj}(y,x),a^{adj}(y,x)) 
        \end{align*}
        \item We interpolate onto the existing grid using $x=a+(1-\delta)(1-f)pd$ to get
        \begin{align*}
            c^{adj}(y,d,a), a^{adj}(y,d,a), d^{adj}(y,d,a)
        \end{align*}
        and value function
        \begin{align*}
            V^{adj}(y,d,a) &= \psi \ln c^{adj}(a,d,y) +(1-\psi) \ln d^{adj}(a,d,y) + \beta W(y',d^{adj}(a,d,y),a^{adj}(a,d,y)) 
        \end{align*}
    \end{itemize}
\end{frame}



\begin{frame}
    \frametitle{Combined Problem}
    \begin{itemize}
        \item Get policy functions by combining adjust and no-adjust solutions.
        \begin{align*}
            c'(y,d,a) &= adjust(a,d,y)c^{adj}(a,d,y) + (1-adjust(a,d,y))c^{noadj}(a,d,y) \\
            a'(y,d,a) &= adjust(a,d,y)a^{adj}(a,d,y) + (1-adjust(a,d,y))a^{noadj}(a,d,y) \\
            d'(y,d,a) &= adjust(a,d,y)d^{adj}(a,d,y) + (1-adjust(a,d,y))d^{noadj}(a,d,y) 
        \end{align*}
    \end{itemize}
\end{frame}
        

%%%%%%%%%%%%%%%%%%%%%%%%%%%%%%%%%%%%%%%%%%%%%%%%%%
\section{Continuous Time}
%%%%%%%%%%%%%%%%%%%%%%%%%%%%%%%%%%%%%%%%%%%%%%%%%%

\begin{frame}
    \frametitle{Continuous Time}
    \begin{itemize}
        \item Alternatively can formulate the problem in continuous time.
        \item Continuous time generally very good for handling stopping time problems, such as when to adjust.
        \item With discretized value function can write problem as
        \begin{align*}
            \min\{\rho v - u(v) - A(v)v, v - v^{*}(v)\} = 0
        \end{align*}
        \begin{itemize}
            \item First term is the no-adjustment utility flow.
            \item Second term is the adjustment choice.
        \end{itemize}
        \item Intuition: if $v < v^{*}(v)$ then do not adjust and get flow value. If adjust $v = v^{*}(v)$ since the flow value $u(v) + A(v)v < \rho v$.
    \end{itemize}
\end{frame}

\begin{frame}
    \frametitle{Considerations}
    \begin{itemize}
        \item To solve continuous time problem see Ben Moll's code on stopping time problems.
        \item In principle continuous time better: do not require sufficiently large noise shocks to make value function differentiable.
        \item But harder to then apply discrete time sequence space methods. Even after aggregation, the distribution of states jumps after a shock so cannot use the fake news algorithm as implemented. (See McKay-Wieland 2021.)
        \item Possible that continuous time SSJ could be used. But need to handle the immediate jumps in the distribution.
    \end{itemize}
\end{frame}

\end{document}